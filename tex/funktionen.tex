\subsection{Allgemeine Mathematische Funktionen}
\renewcommand{\arraystretch}{1.1}
\begin{tabular}{lll}
    \textbf{Symbol} & \textbf{Matlab} & \textbf{Beschreibung}
  \\
     $ ln(x)dx $ &
    \texttt{diff(log(x),x)} &
  \\
    $ |x| $ &
    \texttt{abs(x)} &
    Betrag
  \\
    &
    \texttt{angle(x)} &
    Argument, Phase
  \\
    &  
    \texttt{ceil(x)} & 
    Runden (grösser oder gleich) zur nächsten Zahl
  \\
    &
    \texttt{floor(x)} &
    Runden (kleiner oder gleich) zur nächsten Zahl
  \\
    &
    \texttt{round(x)} &
    Rundung zur nächsten ganzen Zahl 
  \\
    &
    \texttt{conj(x)} &
    Konjugiert komplexe Zahl
  \\
    $ e^x $ &
    \texttt{exp(x)} &
    Exponentioalfunktion 
  \\
    $ ln(x) $ &
    \texttt{log(x)} &
    Natürlicher Logarithmus 
  \\
    $ log(x) $ &
    \texttt{log10(x)} &
    Zehnerlogarithmus
  \\
    &  
    \texttt{imag(x)} &
    Imaginärteil einer Zahl
  \\
    &
    \texttt{real(x)} &
    Realteil einer Zahl
  \\
    $ \frac{x}{y} $ &
    \texttt{rem(x,y)} &
    Ganzzahliger Rest von 
  \\
    $ \sqrt{x} $ &
    \texttt{sqrt(x)} &
    Quadratwurzel
  \\
    $ x! $ &
    \texttt{factorial(x)} &
    Fakultät
  \\
    $ \qquad{n \choose k} $ &
    \texttt{nchoosek(n,k)} &
    Binomialkoeffizient
  \\
    $ X_0 = \mu $ &
    \texttt{mean(Array)} &
    Linearer Mittelwert
  \\
    &
    \texttt{var(Array)} &
    Varianz
  \\
    $ \sigma $ &
    \texttt{std(Array)} &
    Standardabweichung
  \\
    $ X^2 $ &
    \texttt{mean(Array.\texttt{\hoch} 2)} &
    Quadratischer Mittelwert
  \\
    &
    \texttt{[C, D] = xcorr(A, B)} &
    Auto- bzw. Kreuzkorelation
  \\
    $ \frac{2}{\sqrt{\pi}} \cdot \int \limits_0^x e^{-t^2}dt $ &
    \texttt{erf(x)} &
    Errorfunktion
  \\
    $ \frac{2}{\sqrt{\pi}} \cdot \int \limits_x^\infty e^{-t^2}dt = 1 - erf(x) $ &
    \texttt{erfc(x)} &
    Komplementäre Errorfunktion
\end{tabular}