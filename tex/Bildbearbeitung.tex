\subsection{Bildbearbeitung}

\begin{minipage}{0.6\textwidth}
  \begin{tabular}{|p{2.5cm}| p{3.5cm}| p{4cm}|}
  \hline
  Befehl & Beschreibung & Parameter\\ \hline \hline
  imread(filename, fmt) & liest Bilder in Matlab ein& filename = Name das files, fmt = Format \\ \hline
  imwrite(A, filename)& Macht ein neues Bildfile & A = Matlabimage, filename = Name der Zieldatei\\ \hline
  image(C) & Zeigt das Bild an & c = Matrix mit Bildinformationen\\ \hline
  imagesc(C)& zeigt das Bild an, die Bilddaten werden auf den vollen Range aufskaliert & c = Matrix mit Bildinformationen \\ \hline
  imshow(I) & zeigt Bild in neuer Figure an & I = Matlabbild \\ \hline
  imshowpair(A,B)  & Macht ein Bild, welches die Unterschiede zwischen A und B zeigt & A,B = Matlabbilder \\ \hline
  imfinfo(I) & Gibt ein File mit Infos zum Bild aus & I = Bild \\ \hline
  imnoise(I, type)& Fügt dem Bild Rauschen hinzu & I = Matlabbild, type = Type des Rauschens gemäss Tabelle \\ \hline
  im2double(I) & Macht aus einem Bild ein Doublebild & I = Matlabbild \\ \hline
  im2int16(I) & Macht aus einem Bild ein int16 Bild & I = Matlabbild \\ \hline
  im2uint8(I) & Macht aus einem Bild ein uint8 Bild & I = Matlabbild \\ \hline
  im2uint16(I) & Macht aus einem Bild ein uint16 Bild & I = Matlabbild \\ \hline
  imresize(A, scale) & Macht ein neues Bild das scale mal A gross ist & A = Matlabbild, scale = Faktor \\ \hline
  imrotate(A, angle) & Macht ein neues Bild das mit angle rotiert wurde & A = Matlabbild , angle = Winkel in Grad Gegenuhrzeigersinn \\ \hline 
   
  
  
  \end{tabular}
\end{minipage}
\hfill
\begin{minipage}{0.4\textwidth}
\textbf{Unterstütze Formate imread}
\begin{itemize}
\item BMP Windows Bitmap
\item CUR Cursor File
\item GIF Graphics Interchange Format
\item HDF4 Hierarchical Data Format
\item ICO Icon File
\item JPEG Joint Photographic Experts Group
\item JPEG 2000 Joint Photografic Experts Group 2000
\item PBM Portable Bitmap
\item PCX Windows Paintbrush
\item PGM Portable Graymap
\item PNG Portable Network Graphics
\item PPM Portable Pixmap
\item RAS Sun Raster
\item TIFF Tagged Image File Format
\item XWD X Window Dump
\end{itemize}
\textbf{Arten von Noise} \\ 
\begin{tabular}{p{2cm}|p{4cm}}
\hline
gaussian & Weisses Rauschen mit konstantem Mittelwert und Varianz \\ \hline
localvar & Weisses Gaussrauschen mit Mittelwert 0 und Varianzabhängier Intensität \\ \hline
poisson & Poisson Rauschen \\ \hline
salt \& pepper & On and off Pixels \\ \hline
speckle & Multiplicatives Rauschen\\ \hline
\end{tabular}
\\ 
\\
\textbf{Datentypen}

\begin{tabular}{p{2cm}|p{4cm}}
\hline
Type & Name \\
int8 & integer 8bit \\
uint8 & unsigne integer 8bit \\
int16 & integer 16bit \\
uint16 & unsigned integer 16bit\\
double & Double precision real number \\ \hline
\end{tabular}

\end{minipage}