% header.tex von Koeppel Koeppel & Konsortium

%Angaben zum Dokument
\newcommand*\myauthor{Hannes Badertscher}
\newcommand*\mytitle{TI Voyage 200 - Befehlsreferenz}
\newcommand*\mydate{\today}
\newcommand*\myversion{v1.0}

%Schriftgroesse, Layout, Papierformat, Gleichungen Linksbuendig
\documentclass[10pt,twoside,a4paper,fleqn]{article}

%Abmessungen Layout
\usepackage[left=1cm,right=1cm,top=1cm,bottom=1cm,includeheadfoot]{geometry}

%Package fuer Umlaute
\usepackage[utf8]{inputenc}
\usepackage[T1]{fontenc}

%Package fuer Deutsch
\usepackage[ngerman]{babel}

%Formeln
\usepackage{amsmath}
%Spezielle Symbole in Formeln 
\usepackage{amssymb}
%Sonderzeichen
\usepackage{textcomp}
%Bilder und Grafiken
\usepackage{graphicx}
%Textfarben
\usepackage{color}
%Fliesstext
\usepackage{wrapfig}
%Header und Footer
\usepackage{fancyhdr}
%Teile des Dokuments Mehrspaltig
\usepackage{multicol}
%Mehere Zellen vertikal verbinden
\usepackage{multirow} 
%Rotieren von Elementen (Bilder, Tabellen...)
\usepackage{rotating}
%Schriftart
\usepackage[scaled]{helvet}
\renewcommand*\familydefault{\sfdefault}
%Multirow
\usepackage{multirow}
%Einzelne Seite querformat
\usepackage{lscape}


%Seitenaufbau
\pagestyle{fancy} %eigener Seitenstil
%\fancyhf{}

%Linker Header
\lhead{\mytitle}
%Rechter Header
\rhead{\footnotesize{\myversion}}
%Linker Footer
\lfoot{\myauthor}
%Center Footer
\cfoot{\thepage}
%Rechter Footer
\rfoot{\mydate}
%Text Linksbuendig, unregelmaessiger rechter Rand
\raggedright

\renewcommand{\arraystretch}{1.5}
